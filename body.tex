\chapter{Task 1: Basic Probabilities and Visualizations (1)}
The number of meteorites falling on an ocean in a given year can be modelled by one of the following 
distributions. Give a graphic showing the probability of one, two, three… meteorites falling (until the 
probability remains provably less than 0.5\% for any bigger number of meteorites). Calculate the 
expectation and median and show them graphically on this graphic

A Poisson distribution with an expectation of $ \lambda = 16 $

\begin{equation} P\left( X = k \right) = \frac{{e^{ - \lambda } \lambda ^k }}{{k!}}\end{equation}

The probability that one, two, three… meteorites falling can be Calculated as 
P\(X=1\),P\(X=2\),P\(X=3\) until the probability remains less than 0.5\%


\begin{equation} P\left( X = 1 \right) = \frac{{e^{ -16 }.16 ^1 }}{{1!}} =  1.29517596 \times 10^{-11} \end{equation}

\begin{equation} P\left( X = 2 \right) = \frac{{e^{ -16 }.16 ^2 }}{{2!}} =  1.84467854  \times 10^{-9} \end{equation}

\begin{equation} P\left( X = 3 \right) = \frac{{e^{ -16 }.16 ^3 }}{{3!}} =  4.09641181  \times 10^{-8} \end{equation}











\chapter{Task 2: Basic Probabilities and Visualizations (2)}






\chapter{Task 3: Transformed Random Variables}






\chapter{Task 4: Hypothesis Test}






\chapter{Task 5: Regularized Regression }







\chapter{Task 6: Bayesian Estimates}